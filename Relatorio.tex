\documentclass{article}
\usepackage[utf8]{inputenc}
\usepackage[T1]{fontenc}
\usepackage[portuguese]{babel}
\usepackage{amsmath}
\usepackage{graphicx}
\usepackage{hyperref}
\usepackage{geometry}
\geometry{a4paper, margin=1in}

\title{Relatório do Jogo "Phantom Project"}
\author{Nomes dos Integrantes (Preencher aqui)}
\date{\today}

\begin{document}

\maketitle

\begin{abstract}
Este relatório descreve o projeto do jogo "Phantom Project", um jogo de tiro 2D (shoot 'em up) inspirado na série Touhou. O documento aborda a arquitetura do jogo, os requisitos de sistema, instruções para compilação e execução, e uma visão geral de como os principais componentes funcionam.
\end{abstract}

\section{Introdução}
O "Phantom Project" é um jogo de tiro 2D vertical, onde o jogador controla uma personagem que deve desviar de projéteis inimigos e derrotar chefes. O jogo é estruturado em fases, cada uma apresentando desafios e padrões de ataque únicos. A inspiração principal vem da aclamada série de jogos "Touhou Project", conhecida por sua jogabilidade intensa e complexa.

\subsection{Documentação Adicional}
Para uma compreensão mais aprofundada da arquitetura do projeto, todos os diagramas UML detalhados estão disponíveis na pasta \texttt{docs/Diagramas UML/}.

\section{Requisitos do Sistema}
Para compilar e executar o "Phantom Project", os seguintes requisitos de sistema são necessários:
\begin{itemize}
    \item \textbf{IDE NetBeans:} O projeto está configurado para ser aberto e executado diretamente no NetBeans (versão 8.2 ou superior recomendada).
    \item \textbf{Java Development Kit (JDK):} Versão 16 ou superior.
    \item \textbf{Apache Ant:} Ferramenta de automação de build, geralmente incluída no NetBeans.
\end{itemize}

\section{Como Compilar e Executar}
O projeto foi desenvolvido e configurado para o ambiente de desenvolvimento integrado (IDE) NetBeans, que utiliza o Apache Ant para gerenciar o processo de build.

\subsection{Via IDE NetBeans (Método Recomendado)}
A maneira mais simples de compilar e executar o projeto é abri-lo no NetBeans e utilizar os controles da IDE:
\begin{itemize}
    \item \textbf{Executar o Projeto (F6):} Pressionar a tecla F6 (ou clicar no botão "Executar Projeto") irá compilar e iniciar o jogo automaticamente.
    \item \textbf{Limpar e Construir Projeto:} Esta opção no menu "Executar" irá limpar builds anteriores e gerar um novo arquivo JAR executável na pasta \texttt{dist/}.
\end{itemize}

\subsection{Via Linha de Comando (Alternativo)}
Para usuários que preferem a linha de comando, os mesmos comandos do Ant podem ser executados manualmente.

\subsubsection{Compilação}
Para compilar o projeto e gerar o arquivo JAR executável, navegue até o diretório raiz do projeto no terminal e execute o seguinte comando:
\begin{verbatim}
ant clean jar
\end{verbatim}
Este comando irá limpar quaisquer builds anteriores e, em seguida, compilar o código-fonte, empacotando-o em um arquivo JAR localizado na pasta \texttt{dist/}.

\subsubsection{Execução via Ant}
Para executar o jogo diretamente após a compilação, use o comando:
\begin{verbatim}
ant run
\end{verbatim}

\subsubsection{Execução via Arquivo JAR}
Após a compilação, um arquivo JAR executável será gerado em \texttt{dist/PrototipoProjetoPOO.jar}. Você pode executá-lo diretamente usando o Java Runtime Environment (JRE):
\begin{verbatim}
java -jar dist/PrototipoProjetoPOO.jar
\end{verbatim}

\section{Como Funciona (Visão Geral da Arquitetura)}
O "Phantom Project" é construído com uma arquitetura modular, utilizando princípios de Orientação a Objetos para separar as responsabilidades. Os componentes chave incluem:

\begin{itemize}
    \item \textbf{Engine (\texttt{Controler/Engine.java}):} O coração do jogo, responsável pelo loop principal (game loop), gerenciamento do estado do jogo (jogando, pausado, game over) e coordenação entre os outros módulos.
    \item \textbf{Fase (\texttt{Modelo/Fases/Fase.java}) e ScriptDeFase (\texttt{Modelo/Fases/ScriptDeFase.java}):} A classe \texttt{Fase} atua como um contêiner para todos os personagens e elementos de cenário de um nível específico. Cada \texttt{Fase} é controlada por um \texttt{ScriptDeFase} correspondente, que define os eventos, spawns de inimigos e padrões de comportamento para aquele nível.
    \item \textbf{Personagem (\texttt{Modelo/Personagem.java}):} Uma classe abstrata base que define as propriedades e comportamentos comuns a todas as entidades do jogo, como o herói, inimigos, projéteis e itens. Isso permite um alto grau de polimorfismo e extensibilidade.
    \item \textbf{ControleDeJogo (\texttt{Controler/ControleDeJogo.java}) e Quadtree (\texttt{Auxiliar/Personagem/Quadtree.java}):} O \texttt{ControleDeJogo} orquestra a lógica principal do jogo, incluindo a detecção de colisões. Para otimizar este processo, ele utiliza uma \texttt{Quadtree}, uma estrutura de dados espacial que reduz significativamente o número de verificações de colisão necessárias, especialmente em fases com muitos objetos.
    \item \textbf{Cenário (\texttt{Controler/Cenario.java}):} Este painel é responsável por renderizar todos os elementos visuais da fase, incluindo personagens, elementos de fundo e a interface do usuário (HUD).
\end{itemize}

\section{Controles}
Os controles do jogo são baseados no teclado:
\begin{itemize}
    \item \textbf{Setas direcionais / Teclas W, A, S, D:} Movimentação da personagem.
    \item \textbf{Teclas Z / K:} Disparo principal.
    \item \textbf{Teclas X / L:} Ativação de bomba (se disponível).
    \item \textbf{Tecla Shift:} Foco (movimento lento e hitbox visível).
    \item \textbf{Tecla ESC:} Pausar o jogo ou cancelar uma seleção no menu.
    \item \textbf{Tecla Enter:} Selecionar um item no menu.
    \item \textbf{Tecla P:} Salvar o estado atual do jogo.
    \item \textbf{Tecla R:} Carregar um estado de jogo salvo.
    \item \textbf{Tecla R (na tela de Game Over):} Reiniciar o jogo.
    \item \textbf{Combinações de Debug (Modo Debug):}
    \begin{itemize}
        \item \textbf{G + 3:} Ativa/desativa o modo de depuração (exibe hitboxes e FPS).
        \item \textbf{G + 4:} Salva inimigos de teste em arquivos .zip (para funcionalidade de Drag and Drop).
        \item \textbf{G + 5:} Exibe informações de "High Watermark" dos pools de objetos.
    \end{itemize}
    \item \textbf{Drag and Drop (Modo Debug):} Arrastar arquivos .zip de inimigos para a janela do jogo os adicionará na posição do mouse.
\end{itemize}

\end{document}
